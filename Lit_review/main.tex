%Style
\documentclass[12pt]{article}
\usepackage[top=1in, bottom=1in, left=1in, right=1in]{geometry}
\parindent 22pt
\usepackage{fancyhdr}

%Packages
\usepackage{adjustbox}
\usepackage{amsmath}
\usepackage{amsfonts}
\usepackage{amssymb}
\usepackage[english]{babel}
\usepackage{bm}
\usepackage[table]{xcolor}
%\usepackage{tabu}
\usepackage{color,soul}
\usepackage[utf8x]{inputenc}
\usepackage{makecell}
\usepackage{longtable}
\usepackage{multirow}
\usepackage[normalem]{ulem}
\usepackage{etoolbox}
\usepackage{graphicx}
\usepackage{tabularx}
\usepackage{ragged2e}
\usepackage{booktabs}
\usepackage{caption}
\usepackage{fixltx2e}
\usepackage[para, flushleft]{threeparttablex}
\usepackage[capposition=top]{floatrow}
\usepackage{subcaption}
\usepackage{pdfpages}
\usepackage{pdflscape}
\usepackage[sort&compress]{natbib}
\usepackage{bibunits}
\usepackage[colorlinks=true,linkcolor=darkgray,citecolor=darkgray,urlcolor=darkgray,anchorcolor=darkgray]{hyperref}
\usepackage{marvosym}
\usepackage{makeidx}
\usepackage{setspace}
\usepackage{enumerate}
\usepackage{rotating}
\usepackage{epstopdf}
\usepackage[titletoc]{appendix}
\usepackage{framed}
\usepackage{comment}
\usepackage{xr}
\usepackage{titlesec}
\usepackage{footnote}
\usepackage{longtable}
\newlength{\tablewidth}
\setlength{\tablewidth}{9.3in}
\usepackage[bottom]{footmisc}
\usepackage{stackengine}
\newcommand\barbelow[1]{\stackunder[1.2pt]{$#1$}{\rule{1ex}{.085ex}}}
\usepackage{titletoc}
\usepackage{accents}
\usepackage{arydshln }
\usepackage{titletoc}
\titlespacing{\section}{.2pt}{1ex}{1ex}
\setcounter{section}{0}
\renewcommand{\thesection}{\arabic{section}}


\makeatletter
\pretocmd\start@align
{%
  \let\everycr\CT@everycr
  \CT@start
}{}{}
\apptocmd{\endalign}{\CT@end}{}{}
\makeatother
%Watermark
%\usepackage[printwatermark]{xwatermark}
\usepackage{lipsum}
\definecolor{lightgray}{RGB}{220,220,220}
\definecolor{dimgray}{RGB}{105,105,105}

%\newwatermark[allpages,color=lightgray,angle=45,scale=3,xpos=0,ypos=0]{Preliminary Draft}

%Further subsection level
\usepackage{titlesec}
\titleformat{\paragraph}
{\normalfont\normalsize\bfseries}{\theparagraph}{1em}{}
\titlespacing*{\paragraph}
{0pt}{3.25ex plus 1ex minus .2ex}{1.5ex plus .2ex}

\titleformat{\subparagraph}
{\normalfont\normalsize\bfseries}{\thesubparagraph}{1em}{}
\titlespacing*{\subparagraph}
{0pt}{3.25ex plus 1ex minus .2ex}{1.5ex plus .2ex}

%Functions
\DeclareMathOperator{\cov}{Cov}
\DeclareMathOperator{\sign}{sgn}
\DeclareMathOperator{\var}{Var}
\DeclareMathOperator{\plim}{plim}
\DeclareMathOperator*{\argmin}{arg\,min}
\DeclareMathOperator*{\argmax}{arg\,max}

%Math Environments
\usepackage{amsthm}
\newtheoremstyle{mytheoremstyle} % name
    {\topsep}                    % Space above
    {\topsep}                    % Space below
    {\color{black}}                   % Body font
    {}                           % Indent amount
    {\itshape \color{dimgray}}                   % Theorem head font
    {.}                          % Punctuation after theorem head
    {.5em}                       % Space after theorem head
    {}  % Theorem head spec (can be left empty, meaning ?normal?)

\theoremstyle{mytheoremstyle}
\newtheorem{assumption}{Assumption}
\renewcommand\theassumption{\arabic{assumption}}

\theoremstyle{mytheoremstyle}
\newtheorem{assumptiona}{Assumption}
\renewcommand\theassumptiona{\arabic{assumptiona}a}

\newtheorem{assumptionb}{Assumption}
\renewcommand\theassumptionb{\arabic{assumptionb}b}

\newtheorem{assumptionc}{Assumption}
\renewcommand\theassumptionc{\arabic{assumptionc}c}

\theoremstyle{mytheoremstyle}
\newtheorem{lemma}{Lemma}

\theoremstyle{mytheoremstyle}
\newtheorem{proposition}{Proposition}

\theoremstyle{mytheoremstyle}
\newtheorem{corollary}{Corollary}

%Commands
\newcommand\independent{\protect\mathpalette{\protect\independenT}{\perp}}
\def\independenT#1#2{\mathrel{\rlap{$#1#2$}\mkern2mu{#1#2}}}
\newcommand{\overbar}[1]{\mkern 1.5mu\overline{\mkern-1.5mu#1\mkern-1.5mu}\mkern 1.5mu}
\newcommand{\equald}{\ensuremath{\overset{d}{=}}}
\captionsetup[table]{skip=10pt}
%\makeindex

%Table, Figure, and Section Styles
\captionsetup[figure]{labelfont={bf},name={Figure},labelsep=period}
\renewcommand{\thefigure}{\arabic{figure}}
\captionsetup[table]{labelfont={bf},name={Table},labelsep=period}
\renewcommand{\thetable}{\arabic{table}}
\titleformat{\section}{\centering \normalsize \bf}{\thesection.}{0em}{}%\titlespacing*{\subsection}{0pt}{0\baselineskip}{0\baselineskip}
\renewcommand{\thesection}{\arabic{section}}

\titleformat{\subsection}{\flushleft \normalsize \bf}{\thesubsection}{0em}{}
\renewcommand{\thesubsection}{\arabic{section}.\arabic{subsection}}

%No indent
\setlength\parindent{24pt}
\setlength{\parskip}{5pt}

%Logo
%\AddToShipoutPictureBG{%
%  \AtPageUpperLeft{\raisebox{-\height}{\includegraphics[width=1.5cm]{uchicago.png}}}
%}

\newcolumntype{L}[1]{>{\raggedright\let\newline\\\arraybackslash\hspace{0pt}}m{#1}}
\newcolumntype{C}[1]{>{\centering\let\newline\\\arraybackslash\hspace{0pt}}m{#1}}
\newcolumntype{R}[1]{>{\raggedleft\let\newline\\\arraybackslash\hspace{0pt}}m{#1}} 

\newcommand{\mr}{\multirow}
\newcommand{\mc}{\multicolumn}

%\newcommand{\comment}[1]{}


\title{Literature Review-ECON 635}
\author{Onur Karagozoglu}
\date{}
\doublespacing

\begin{document}

\maketitle


A substantial portion of the literature in economics has examined the link between poverty, economic stability, and marriage formation, yet the causal mechanisms and direction of the relationship between economic well-being and marriage remain unsettled. As we discussed in class, the dominant U.S. evidence suggests that economic disadvantage depresses marriage, especially in the most disadvantaged portions of society. Foundational and seminal works such as \citet{Wilson1987} and \citet{Murray1994} attribute the ``retreat from marriage'' among low-income populations to the disappearance of stable employment and the consequent erosion of the economic foundations of household formation. Ethnographic evidence in \citet{EdinKefalas2005} and \citet{Cherlin2004} reinforces this narrative, portraying marriage as a ``capstone'' achieved only after financial security and depicting poor individuals as valuing marriage but deferring it due to its symbolic and, especially, financial costs. In this view, poverty and precarious labor markets make marriage almost a luxury good, primarily accessible to the middle and upper classes. 

This interpretation from previous and seminal works is supported by empirical studies. As \citet{Autor2019} show, declines in male manufacturing employment reduced marriage rates among less educated men, highlighting the role of male earnings capacity in family formation. \citet{GouldPaserman2003} similarly find that local labor market shocks influence marriage and divorce patterns by altering the economic prospects of potential spouses. \citet{LundbergPollak2007} conceptualize the ``uneven retreat from marriage'' as a class-based phenomenon driven by declining economic returns to specialization within marriage and the rise of cohabitation. A more recent study by \citet{BlauKahnWaldfogel2013} documents how the economic polarization of the family has widened over time, with stable marriage increasingly concentrated among higher-educated and higher-income groups rather than in the disadvantaged segments of society. Lastly, as we discussed in class, recent contributions by \citet{GarciaHeckman2023} demonstrate these findings within a broader model of intergenerational inequality. Using longitudinal U.S. data, they show that disadvantaged groups, those with low skills and early, life socioeconomic constraints—display persistently lower marriage rates and family stability. Up to this point, it is clearly confirmed that, especially in the U.S. context, poverty and economic instability reinforce each other in a self-perpetuating cycle that suppresses marriage formation.

Yet this study suggests alternative theoretical perspectives, arguing that the relationship between poverty and marriage may not be uniformly negative. In this work, we consider different views such as interpreting marriage as a form of social insurance, especially where formal credit and insurance markets are incomplete. As \citet{RosenzweigStark1989} and \citet{Townsend1994} demonstrate in developing contexts, marriage and kinship networks serve as substitutes for missing markets, allowing risk pooling and consumption smoothing across households. Extending this intuition to advanced economies, households facing high income volatility may benefit from pooling two uncertain incomes into one. In this framework, marriage provides insurance rather than a consumption cost, implying that low-income households may have stronger, not weaker, incentives to marry when risk-sharing dominates financial cost. This mechanism can create an incentive to marry, particularly among the most disadvantaged groups in society. 

Institutional structures, especially the design of tax and benefit systems, play a crucial role in shaping how income influences marriage decisions. In the United States, programs such as the Earned Income Tax Credit (EITC) have generated both ``marriage penalties'' and ``marriage bonuses,'' depending on how household earnings are distributed between partners \citep{EissaHoynes2004}. For dual-earner low-income couples, joint filing can reduce net after-tax income and thereby create a disincentive to marry. On the other hand, for single-earner or highly asymmetric-earning households, the EITC can increase disposable income upon marriage, effectively acting as a subsidy that creates an incentive to marry. These opposing effects make the U.S. case particularly controversial: while some studies emphasize that complex benefit interactions have eroded the financial appeal of marriage among the poor \citep{EllwoodJencks2004}, others point out that the same tax credit can raise household resources for disadvantaged single-earner families, potentially encouraging formal unions and stability \citep{Hoynes2015, Michelmore2018}. As the U.S. system combines both penalties and bonuses within a single policy framework, it offers an ambiguous empirical picture of how fiscal design interacts with marriage incentives. To better understand the role of institutional architecture, it is informative to compare the U.S. with other developed economies where policy incentives are more clearly structured. In Germany, for instance, \citet{Steiner2011} show that the joint taxation system (\textit{Ehegattensplitting}) disproportionately benefits low-income and single-earner couples, creating a direct fiscal incentive to marry. Similarly, \citet{DoepkeKindermann2019} demonstrate that child allowance and parental leave reforms have measurable positive effects on family formation decisions. Examining the effects of these contrasting institutional contexts on marriage allows us to identify how the design of tax and transfer systems can transform the financial calculus of marriage, from a potential deterrent in one setting to a clear incentive in another, and highlights the importance of comparing fiscal structures across advanced economies when evaluating the poverty–marriage relationship.

Cultural and ideological norms further complicate the relationship between income and marriage, adding one more layer to the ``marriage puzzle.'' \citet{Lesthaeghe2010, Lesthaeghe2014} and \citet{Cherlin2004} describe the ``second demographic transition'' and the ``deinstitutionalization of marriage'' as processes driven by secularization, individualism, and changing gender roles. These trends are strongest in affluent, liberal societies. However, in more conservative or religious environments, often overrepresented among lower-income groups, marriage continues to hold strong symbolic and moral value. \citet{GlaeserSacerdote2007} and \citet{McLanahanPercheski2008} note that cultural and community norms can sustain higher marriage rates despite economic constraints. The persistence of such normative frameworks means that identical economic conditions can yield opposite behavioral responses depending on cultural and ideological context, which is an important point to emphasize. 

Taken together, this study provides a multidimensional, context-dependent, and layered relationship between economic status and marriage. The U.S. evidence, exemplified by \citet{GarciaHeckman2023} and the broader ``marriage gap'' tradition, indicates that economic instability and low income tend to suppress marriage through high costs and uncertainty. Yet the comparative and theoretical evidence we propose identifies mechanisms such as social insurance, fiscal design, and cultural norms through which poverty may also enhance marriage incentives. The coexistence of these opposing effects constitutes a central empirical puzzle: the same level of economic disadvantage can either hinder or foster marriage depending on institutional, policy, and cultural environments. Understanding which mechanism dominates under which conditions is therefore crucial for explaining cross-country variation in family formation and for designing effective policies to support family stability and demographic sustainability.
\newpage
\bibliographystyle{apalike}
\bibliography{reference4}

\end{document}


