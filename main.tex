%Style
\documentclass[12pt]{article}
\usepackage[top=1in, bottom=1in, left=1in, right=1in]{geometry}
\parindent 22pt
\usepackage{fancyhdr}

%Packages
\usepackage{adjustbox}
\usepackage{amsmath}
\usepackage{amsfonts}
\usepackage{amssymb}
\usepackage[english]{babel}
\usepackage{bm}
\usepackage[table]{xcolor}
%\usepackage{tabu}
\usepackage{color,soul}
\usepackage[utf8x]{inputenc}
\usepackage{makecell}
\usepackage{longtable}
\usepackage{multirow}
\usepackage[normalem]{ulem}
\usepackage{etoolbox}
\usepackage{graphicx}
\usepackage{tabularx}
\usepackage{ragged2e}
\usepackage{booktabs}
\usepackage{caption}
\usepackage{fixltx2e}
\usepackage[para, flushleft]{threeparttablex}
\usepackage[capposition=top]{floatrow}
\usepackage{subcaption}
\usepackage{pdfpages}
\usepackage{pdflscape}
\usepackage[sort&compress]{natbib}
\usepackage{bibunits}
\usepackage[colorlinks=true,linkcolor=darkgray,citecolor=darkgray,urlcolor=darkgray,anchorcolor=darkgray]{hyperref}
\usepackage{marvosym}
\usepackage{makeidx}
\usepackage{setspace}
\usepackage{enumerate}
\usepackage{rotating}
\usepackage{epstopdf}
\usepackage[titletoc]{appendix}
\usepackage{framed}
\usepackage{comment}
\usepackage{xr}
\usepackage{titlesec}
\usepackage{footnote}
\usepackage{longtable}
\newlength{\tablewidth}
\setlength{\tablewidth}{9.3in}
\usepackage[bottom]{footmisc}
\usepackage{stackengine}
\newcommand\barbelow[1]{\stackunder[1.2pt]{$#1$}{\rule{1ex}{.085ex}}}
\usepackage{titletoc}
\usepackage{accents}
\usepackage{arydshln }
\usepackage{titletoc}
\titlespacing{\section}{.2pt}{1ex}{1ex}
\setcounter{section}{0}
\renewcommand{\thesection}{\arabic{section}}


\makeatletter
\pretocmd\start@align
{%
  \let\everycr\CT@everycr
  \CT@start
}{}{}
\apptocmd{\endalign}{\CT@end}{}{}
\makeatother
%Watermark
%\usepackage[printwatermark]{xwatermark}
\usepackage{lipsum}
\definecolor{lightgray}{RGB}{220,220,220}
\definecolor{dimgray}{RGB}{105,105,105}

%\newwatermark[allpages,color=lightgray,angle=45,scale=3,xpos=0,ypos=0]{Preliminary Draft}

%Further subsection level
\usepackage{titlesec}
\titleformat{\paragraph}
{\normalfont\normalsize\bfseries}{\theparagraph}{1em}{}
\titlespacing*{\paragraph}
{0pt}{3.25ex plus 1ex minus .2ex}{1.5ex plus .2ex}

\titleformat{\subparagraph}
{\normalfont\normalsize\bfseries}{\thesubparagraph}{1em}{}
\titlespacing*{\subparagraph}
{0pt}{3.25ex plus 1ex minus .2ex}{1.5ex plus .2ex}

%Functions
\DeclareMathOperator{\cov}{Cov}
\DeclareMathOperator{\sign}{sgn}
\DeclareMathOperator{\var}{Var}
\DeclareMathOperator{\plim}{plim}
\DeclareMathOperator*{\argmin}{arg\,min}
\DeclareMathOperator*{\argmax}{arg\,max}

%Math Environments
\usepackage{amsthm}
\newtheoremstyle{mytheoremstyle} % name
    {\topsep}                    % Space above
    {\topsep}                    % Space below
    {\color{black}}                   % Body font
    {}                           % Indent amount
    {\itshape \color{dimgray}}                   % Theorem head font
    {.}                          % Punctuation after theorem head
    {.5em}                       % Space after theorem head
    {}  % Theorem head spec (can be left empty, meaning ?normal?)

\theoremstyle{mytheoremstyle}
\newtheorem{assumption}{Assumption}
\renewcommand\theassumption{\arabic{assumption}}

\theoremstyle{mytheoremstyle}
\newtheorem{assumptiona}{Assumption}
\renewcommand\theassumptiona{\arabic{assumptiona}a}

\newtheorem{assumptionb}{Assumption}
\renewcommand\theassumptionb{\arabic{assumptionb}b}

\newtheorem{assumptionc}{Assumption}
\renewcommand\theassumptionc{\arabic{assumptionc}c}

\theoremstyle{mytheoremstyle}
\newtheorem{lemma}{Lemma}

\theoremstyle{mytheoremstyle}
\newtheorem{proposition}{Proposition}

\theoremstyle{mytheoremstyle}
\newtheorem{corollary}{Corollary}

%Commands
\newcommand\independent{\protect\mathpalette{\protect\independenT}{\perp}}
\def\independenT#1#2{\mathrel{\rlap{$#1#2$}\mkern2mu{#1#2}}}
\newcommand{\overbar}[1]{\mkern 1.5mu\overline{\mkern-1.5mu#1\mkern-1.5mu}\mkern 1.5mu}
\newcommand{\equald}{\ensuremath{\overset{d}{=}}}
\captionsetup[table]{skip=10pt}
%\makeindex

%Table, Figure, and Section Styles
\captionsetup[figure]{labelfont={bf},name={Figure},labelsep=period}
\renewcommand{\thefigure}{\arabic{figure}}
\captionsetup[table]{labelfont={bf},name={Table},labelsep=period}
\renewcommand{\thetable}{\arabic{table}}
\titleformat{\section}{\centering \normalsize \bf}{\thesection.}{0em}{}%\titlespacing*{\subsection}{0pt}{0\baselineskip}{0\baselineskip}
\renewcommand{\thesection}{\arabic{section}}

\titleformat{\subsection}{\flushleft \normalsize \bf}{\thesubsection}{0em}{}
\renewcommand{\thesubsection}{\arabic{section}.\arabic{subsection}}

%No indent
\setlength\parindent{24pt}
\setlength{\parskip}{5pt}

%Logo
%\AddToShipoutPictureBG{%
%  \AtPageUpperLeft{\raisebox{-\height}{\includegraphics[width=1.5cm]{uchicago.png}}}
%}

\newcolumntype{L}[1]{>{\raggedright\let\newline\\\arraybackslash\hspace{0pt}}m{#1}}
\newcolumntype{C}[1]{>{\centering\let\newline\\\arraybackslash\hspace{0pt}}m{#1}}
\newcolumntype{R}[1]{>{\raggedleft\let\newline\\\arraybackslash\hspace{0pt}}m{#1}} 

\newcommand{\mr}{\multirow}
\newcommand{\mc}{\multicolumn}

%\newcommand{\comment}[1]{}


\title{Research Proposal-ECON635}
\author{Onur Karagozoglu}
\date{}

\begin{document}

\maketitle

\section*{Research Question}

\textbf{Question:} Reforming the Social Security Taxable Maximum and Its Implications for Redistribution and Efficiency

This project examines how alternative reforms to the Social Security taxable maximum would affect both the distribution of tax burdens and the efficiency of economic activity and to map out how different policy designs alter outcomes for redistribution, efficiency, and fiscal sustainability.  

\section*{Motivation and Contribution}

The Social Security taxable maximum (``wage cap'') has been a defining element of the U.S. retirement system since 1937. For decades the cap was adjusted through ad hoc congressional action until the 1972 amendments linked it to the National Average Wage Index (NAWI) \citep{SSA2024}. Since then, the cap has risen automatically from under \$10{,}000 in the early 1970s to \$176{,}100 in 2025. While indexation preserves the relationship between the cap and average earnings, it leaves unresolved the normative policy debate over whether the cap should be raised further, eliminated, or redesigned.  

This debate has intensified as demographic change places mounting pressure on Social Security. Especially after the worker-to-beneficiary ratio has declined from roughly five in 1960 to less than three today, and is projected to fall close to two within two decades \citep{SSATrustees2024}. Without a proper reform, with the effect of the aging population and unattainability of the Trust Fund, is expected to be depleted in the 2030s. In this environment, the taxable maximum is central to ongoing discussions, with proposals ranging from eliminating the cap \citep{CRS2021}, to substantially raising it \citep{PGPF2025}, to ``donut-hole'' approaches that tax only very high incomes \citep{CBO2015}.  

Before my contribution it will be meaningful to mention about previous research briefly. \citet{Bagchi2017} uses an overlapping-generations model to show that eliminating the cap raises revenue but lowers output and welfare. Think-tank simulations \citep{Heritage2005,Manhattan2024} emphasize adverse labor-supply and saving effects. Yet most of the existing literature evaluates isolated scenarios or relies on actuarial projections rather than embedding reforms into a common macroeconomic framework and evaluating them.  


\textbf{Contribution.} This project offers a systematic macroeconomic analysis of how reforms to the Social Security taxable maximum affect redistribution and efficiency. Rather than evaluating a single scenario, it embeds alternative policy designs such as raising the cap, donut-hole taxation, or shifting contributions above the cap to the income-tax margin within one coherent framework. The goal is to clarify the mechanisms, such as labor-supply responses at the top, benefit-credit feedbacks, and incidence differences, that drive the outcomes.  

This study is worthwhile as reforms can have multiple effects, particularly on the equity–efficiency trade-off, and may generate unexpected consequences that deserve careful examination. On the equity side, higher contributions from top earners may appear redistributive, but because Social Security benefits are tied to lifetime taxable earnings, additional contributions can translate into higher future benefits. This raises the question of whether inequality is truly reduced or merely shifted across time. On the efficiency side, the impact of higher marginal taxes depends on which groups are affected and how they adjust—high earners may respond differently, with stronger avoidance margins such as shifting income forms, and firms may adjust wage–capital shares in general equilibrium. These considerations mean that the trade-off between redistribution and efficiency is not mechanical, and its slope and shape depend on the specific reform design.  

The contribution has a potential for a particular direction of results. Even if findings confirm the intuition that broader tax bases reduce inequality at some efficiency cost, the value lies in quantifying magnitudes, decomposing mechanisms, and showing how different designs position themselves relative to one another. In this way, the project bridges the gap between actuarial revenue scoring and macroeconomic analysis of Social Security reform.  


\section*{Data}

The analysis requires information on earnings relative to the taxable maximum, the share of workers above the cap, and revenues collected. Publicly available datasets are sufficient for an initial exploration.  

The Current Population Survey (CPS) and the Panel Study of Income Dynamics (PSID) provide micro-level data on earnings, hours, and demographics, allowing estimation of how many workers fall above or below the cap. The Survey of Consumer Finances (SCF) adds detail on the very top of the income distribution. On the aggregate side, the Social Security Administration (SSA) publishes annual statistics on the taxable maximum, the proportion of covered earnings subject to contributions, and Trust Fund projections. The National Average Wage Index (NAWI) documents the basis for historical adjustments to the cap.  

These sources allow calibration of the earnings distribution, illustration of how reforms expand the tax base, and simulations of redistribution and efficiency implications. For further research, restricted administrative datasets such as the SSA Detailed Earnings Records or IRS-linked panels could provide precise measurement of top earners and benefit accruals, though these are not essential for the present stage.  


\newpage
\bibliographystyle{apalike}
\bibliography{reference}

\end{document}


