%Style
\documentclass[12pt]{article}
\usepackage[top=1in, bottom=1in, left=1in, right=1in]{geometry}
\parindent 22pt
\usepackage{fancyhdr}

%Packages
\usepackage{adjustbox}
\usepackage{amsmath}
\usepackage{amsfonts}
\usepackage{amssymb}
\usepackage[english]{babel}
\usepackage{bm}
\usepackage[table]{xcolor}
%\usepackage{tabu}
\usepackage{color,soul}
\usepackage[utf8x]{inputenc}
\usepackage{makecell}
\usepackage{longtable}
\usepackage{multirow}
\usepackage[normalem]{ulem}
\usepackage{etoolbox}
\usepackage{graphicx}
\usepackage{tabularx}
\usepackage{ragged2e}
\usepackage{booktabs}
\usepackage{caption}
\usepackage{fixltx2e}
\usepackage[para, flushleft]{threeparttablex}
\usepackage[capposition=top]{floatrow}
\usepackage{subcaption}
\usepackage{pdfpages}
\usepackage{pdflscape}
\usepackage[sort&compress]{natbib}
\usepackage{bibunits}
\usepackage[colorlinks=true,linkcolor=darkgray,citecolor=darkgray,urlcolor=darkgray,anchorcolor=darkgray]{hyperref}
\usepackage{marvosym}
\usepackage{makeidx}
\usepackage{setspace}
\usepackage{enumerate}
\usepackage{rotating}
\usepackage{epstopdf}
\usepackage[titletoc]{appendix}
\usepackage{framed}
\usepackage{comment}
\usepackage{xr}
\usepackage{titlesec}
\usepackage{footnote}
\usepackage{longtable}
\newlength{\tablewidth}
\setlength{\tablewidth}{9.3in}
\usepackage[bottom]{footmisc}
\usepackage{stackengine}
\newcommand\barbelow[1]{\stackunder[1.2pt]{$#1$}{\rule{1ex}{.085ex}}}
\usepackage{titletoc}
\usepackage{accents}
\usepackage{arydshln }
\usepackage{titletoc}
\titlespacing{\section}{.2pt}{1ex}{1ex}
\setcounter{section}{0}
\renewcommand{\thesection}{\arabic{section}}


\makeatletter
\pretocmd\start@align
{%
  \let\everycr\CT@everycr
  \CT@start
}{}{}
\apptocmd{\endalign}{\CT@end}{}{}
\makeatother
%Watermark
%\usepackage[printwatermark]{xwatermark}
\usepackage{lipsum}
\definecolor{lightgray}{RGB}{220,220,220}
\definecolor{dimgray}{RGB}{105,105,105}

%\newwatermark[allpages,color=lightgray,angle=45,scale=3,xpos=0,ypos=0]{Preliminary Draft}

%Further subsection level
\usepackage{titlesec}
\titleformat{\paragraph}
{\normalfont\normalsize\bfseries}{\theparagraph}{1em}{}
\titlespacing*{\paragraph}
{0pt}{3.25ex plus 1ex minus .2ex}{1.5ex plus .2ex}

\titleformat{\subparagraph}
{\normalfont\normalsize\bfseries}{\thesubparagraph}{1em}{}
\titlespacing*{\subparagraph}
{0pt}{3.25ex plus 1ex minus .2ex}{1.5ex plus .2ex}

%Functions
\DeclareMathOperator{\cov}{Cov}
\DeclareMathOperator{\sign}{sgn}
\DeclareMathOperator{\var}{Var}
\DeclareMathOperator{\plim}{plim}
\DeclareMathOperator*{\argmin}{arg\,min}
\DeclareMathOperator*{\argmax}{arg\,max}

%Math Environments
\usepackage{amsthm}
\newtheoremstyle{mytheoremstyle} % name
    {\topsep}                    % Space above
    {\topsep}                    % Space below
    {\color{black}}                   % Body font
    {}                           % Indent amount
    {\itshape \color{dimgray}}                   % Theorem head font
    {.}                          % Punctuation after theorem head
    {.5em}                       % Space after theorem head
    {}  % Theorem head spec (can be left empty, meaning ?normal?)

\theoremstyle{mytheoremstyle}
\newtheorem{assumption}{Assumption}
\renewcommand\theassumption{\arabic{assumption}}

\theoremstyle{mytheoremstyle}
\newtheorem{assumptiona}{Assumption}
\renewcommand\theassumptiona{\arabic{assumptiona}a}

\newtheorem{assumptionb}{Assumption}
\renewcommand\theassumptionb{\arabic{assumptionb}b}

\newtheorem{assumptionc}{Assumption}
\renewcommand\theassumptionc{\arabic{assumptionc}c}

\theoremstyle{mytheoremstyle}
\newtheorem{lemma}{Lemma}

\theoremstyle{mytheoremstyle}
\newtheorem{proposition}{Proposition}

\theoremstyle{mytheoremstyle}
\newtheorem{corollary}{Corollary}

%Commands
\newcommand\independent{\protect\mathpalette{\protect\independenT}{\perp}}
\def\independenT#1#2{\mathrel{\rlap{$#1#2$}\mkern2mu{#1#2}}}
\newcommand{\overbar}[1]{\mkern 1.5mu\overline{\mkern-1.5mu#1\mkern-1.5mu}\mkern 1.5mu}
\newcommand{\equald}{\ensuremath{\overset{d}{=}}}
\captionsetup[table]{skip=10pt}
%\makeindex

%Table, Figure, and Section Styles
\captionsetup[figure]{labelfont={bf},name={Figure},labelsep=period}
\renewcommand{\thefigure}{\arabic{figure}}
\captionsetup[table]{labelfont={bf},name={Table},labelsep=period}
\renewcommand{\thetable}{\arabic{table}}
\titleformat{\section}{\centering \normalsize \bf}{\thesection.}{0em}{}%\titlespacing*{\subsection}{0pt}{0\baselineskip}{0\baselineskip}
\renewcommand{\thesection}{\arabic{section}}

\titleformat{\subsection}{\flushleft \normalsize \bf}{\thesubsection}{0em}{}
\renewcommand{\thesubsection}{\arabic{section}.\arabic{subsection}}

%No indent
\setlength\parindent{24pt}
\setlength{\parskip}{5pt}

%Logo
%\AddToShipoutPictureBG{%
%  \AtPageUpperLeft{\raisebox{-\height}{\includegraphics[width=1.5cm]{uchicago.png}}}
%}

\newcolumntype{L}[1]{>{\raggedright\let\newline\\\arraybackslash\hspace{0pt}}m{#1}}
\newcolumntype{C}[1]{>{\centering\let\newline\\\arraybackslash\hspace{0pt}}m{#1}}
\newcolumntype{R}[1]{>{\raggedleft\let\newline\\\arraybackslash\hspace{0pt}}m{#1}} 

\newcommand{\mr}{\multirow}
\newcommand{\mc}{\multicolumn}

%\newcommand{\comment}[1]{}


\title{Research Proposal 3-ECON 635}
\author{Onur Karagozoglu}
\date{}
\doublespacing

\begin{document}

\maketitle

\section*{Research Question}

\textbf{Question:} Does poverty deter marriage by raising its costs, or can low income under certain institutional and cultural settings actually increase the incentives to marry?

This project examines the puzzling relationship between income and marriage. As discussed in class, while U.S. evidence shows that disadvantaged groups marry less due to marriage costs and the instability of poverty, alternative mechanisms such as household income pooling, tax and child subsidies, and conservative social norms suggest that poverty may in some cases strengthen incentives to marry. This tension raises a broader question: under what conditions does low income suppress, and under what conditions does it foster, marriage formation?

\section*{Motivation and Contribution}

The motivation for this project stems from a fundamental puzzle in the economics of family formation. As discussed in the class, clear evidence from \citet{GarciaHeckman2023} demonstrates that disadvantaged groups in the U.S. have lower marriage rates. In addition to this discussion, a well-established literature also argues that poverty depresses marriage. Seminal contributions such as \citet{Wilson1987} and \citet{Murray1994} describe how the disappearance of stable employment opportunities eroded the economic foundations of marriage among the poor in the U.S. More recent ethnographic and quantitative work by \citet{EdinKefalas2005}, \citet{Cherlin2004}, \citet{Autor2019} and \citet{GarciaHeckman2023} demonstrates that disadvantaged groups—especially Black Americans and those at the bottom of the income distribution—are less likely to marry, with explanations pointing to the high costs of ceremonies, the economic risks of instability, and the perception that marriage requires a financial “arrival.” In this perspective, poverty and precarious labor markets make marriage less accessible due to its high costs, and the institution becomes increasingly concentrated among the middle and upper classes. However, this dominant view sits uneasily with alternative theoretical and empirical insights that point in the opposite direction.

First, marriage can be understood and seen as a social insurance device. In the absence of well-functioning credit and insurance markets, pooling resources through household formation provides protection against income shocks, unemployment, and childrearing costs. As \citet{RosenzweigStark1989} and \citet{Townsend1994} show in developing contexts, family formation often substitutes for missing financial markets especially in the disadvantaged groups. Transposed to advanced economies, this suggests that for poor households, the gains from dual incomes in one household and shared risk may be greater, not smaller.

Second, institutional incentives also have the potential to alter the status-quo. In many welfare states, especially European countries with high tax and transfer systems reward marriage and childbearing, particularly at the lower end of the income distribution. \citet{Steiner2011} document how Germany’s joint taxation system disproportionately benefits low-income households, while \citet{DoepkeKindermann2019} show that child subsidies shift family formation decisions, especially among poorer families. In such contexts, the marginal financial gain from marrying may outweigh the costs, in sharp contrast to the U.S.

Third, cultural and ideological factors complicate the narrative and make the puzzle even more interesting. Conservative or religious groups, which are often overrepresented among poorer populations, attach higher symbolic and moral value to marriage. Studies such as \citet{Glaeser2007} and \citet{Lesthaeghe2010} suggest that cultural norms can sustain higher marriage propensities among the disadvantaged, even when resources are limited, as a result an increasing the incentive to marry which is exactly opposite case to the U.S.

The coexistence of these perspectives creates an interesting puzzle. On one hand, the dominant U.S.-based empirical literature demonstrates that poverty depresses marriage, as discussed in class and in the literature. On the other hand, my arguments above, theory and cross-national evidence indicate mechanisms through which poverty may actually encourage household formation, especially among the most disadvantaged groups.

The motivation behind this question is also to contribute to a prominent policy debate. Even in modern times, marriage remains a central institution linking family formation with fertility. Advanced economies face unprecedented demographic challenges such as fertility rates below replacement, aging populations, and the looming fiscal strain of supporting growing retiree cohorts with shrinking working-age populations \citep{BloomCanningFink2010, DoepkeTertilt2016}. Understanding how poverty shapes marriage decisions is therefore crucial for evaluating demographic policy. If poor segments of society are less likely to marry because of the marriage cost barriers, fertility-enhancing policies may need to address income insecurity and upfront expenses. If, instead, poor households marry more when fiscal incentives align, then pro-natalist policies may be especially effective when targeted toward disadvantaged groups. Either way, resolving or even questioning this puzzle helps to demonstrate the dynamics behind aging populations and low birth rates and provides guidance for effective policy reforms.

\noindent \textbf{Relation to Literature.}
The project is positioned at the intersection of three literatures. First, the U.S. “marriage gap” tradition emphasizes economic instability as a deterrent to marriage \citep{Wilson1987, Murray1994, EdinKefalas2005, Autor2019, GarciaHeckman2023}. Second, the family policy literature in Europe highlights how tax incentives and subsidies shape household formation \citep{Steiner2011, EllwoodJencks2004, DoepkeKindermann2019}. Third, cultural sociology emphasizes conservative values as marriage-promoting, even under economic strain \citep{Cherlin2004, Lesthaeghe2010}. Each perspective captures part of the story, but none integrates them into a framework that explains when poverty reduces and when it increases marriage. This project addresses that gap, contributing to the discussion of how family subsidies can be more effectively designed.

\section*{Contribution}

This project suggests a contribution the literature in three ways:

\begin{enumerate}
\item \textbf{Theoretical.} Develop a unified framework in which marriage has dual roles: a costly commitment good and a household insurance mechanism. This framing clarifies the conditions under which poverty suppresses marriage (when costs dominate) versus encourages it (when risk-sharing, subsidies, or norms dominate).
\item \textbf{Empirical.} Conduct a comparative study of the U.S., with extensions to other OECD countries, to test how the poverty–marriage relationship depends on fiscal and cultural environments. Identification strategies will exploit quasi-experimental policy reforms such as U.S. Earned Income Tax Credit expansions.
\item \textbf{Policy relevance.} Connect the micro-level determinants of marriage among the poor to macro-level challenges of aging and fertility decline. The results will inform whether family subsidies and tax reforms can effectively stimulate household formation in disadvantaged groups, as a result contributing to demographic sustainability and directing the correct policies to its target audience.
\end{enumerate}

\section*{Data}

The empirical strategy relies on rich microdata and cross-country sources:

\begin{itemize}
\item \textbf{United States.} Panel Study of Income Dynamics (PSID), Current Population Survey (CPS), and Survey of Income and Program Participation (SIPP) provide longitudinal data on marriage, fertility, income, and program participation.
\item \textbf{Cross-country comparison.} OECD Family Database, World Values Survey (WVS), and European Social Survey (ESS) capture institutional variation and cultural attitudes toward marriage.
\item \textbf{Policy reforms.} U.S. Earned Income Tax Credit (EITC) expansions in the 1990s, Germany’s 2007 Elterngeld parental leave reform, and child allowance reforms in Nordic countries provide quasi-experimental variation to study causal effects of income and subsidies on marriage.
\end{itemize}

These sources allow the project to test systematically whether poverty deters or encourages marriage, depending on institutional and cultural environments. The findings will deepen our understanding of how inequality interacts with family formation and provide critical insights for demographic policy in the context of aging populations.

\bigskip 

\noindent \textbf{[JLG: Your best proposal yet. Concrete, properly motivated, and policy relevant. Good use of Git and \LaTeX.]}

\vspace{3em}

\noindent \textbf{[AZ: I am intrigued and could see this going in several directions. I think you can also explore policies around incarceration in the U.S. (e.g., California's Three Strikes law). As you know these disproportionately affect lower-income and non-white families, but it also may affect the marriageability of young men. It could provide additional sources for quasi-experimental design.]}

\newpage
\bibliographystyle{apalike}
\bibliography{reference3}

\end{document}


