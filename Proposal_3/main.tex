\input{preamble.tex}

\title{Research Proposal 3-ECON 635}
\author{Onur Karagozoglu}
\date{}
\doublespacing

\begin{document}

\maketitle

\section*{Research Question}

\textbf{Question:} Does poverty deter marriage by raising its costs, or can low income under certain institutional and cultural settings actually increase the incentives to marry?

This project examines the puzzling relationship between income and marriage. As we discussed in the class, while U.S. evidence shows that disadvantaged groups marry less due to the marriage costs and instability of poverty, alternative mechanisms such as household income pooling, tax and child subsidies, and conservative social norms suggest that poverty may in some cases strengthen incentives to marry. This tension raises a broader question: under what conditions does low income suppress, and under what conditions does it foster, marriage formation?

\section*{Motivation and Contribution}

The motivation for this project stems from a fundamental puzzle in the economics of family formation. As we discussed in the class, a clear evidence from \citet{GarciaHeckman2023} demonstrates that disadvantaged groups in the U.S. have lower rate of marriage. In addition to our discussion in the class, a well-structured literature argues also that poverty depresses marriage. Seminal contributions such as \citet{Wilson1987} and \citet{Murray1994} describe how the disappearance of stable employment opportunities eroded the economic foundations of marriage among the poor in the U.S. More recent ethnographic and quantitative work by \citet{EdinKefalas2005}, \citet{Cherlin2004}, and \citet{Autor2019} demonstrates that disadvantaged groups—especially Black Americans and those in the bottom of the income distribution—are less likely to marry, with explanations pointing to the high costs of ceremonies, the economic risks of instability, and the perception that marriage requires a financial “arrival.” In this perspective, poverty and precarious labor markets make marriage less accessible, and the institution becomes increasingly concentrated among the middle and upper classes.

However, the puzzling part starting from here and this dominant view sits uneasily with alternative theoretical and empirical insights that point in the opposite direction.

First of  all, marriage can be understood and seen as a social insurance device. In the absence of well-functioning credit and insurance markets, pooling resources through household formation provides protection against income shocks, unemployment, and childrearing costs. As \citet{RosenzweigStark1989} and \citet{Townsend1994} show in developing contexts, family formation often substitutes for missing financial markets. Transposed to advanced economies, this suggests that for poor households, the gains from dual incomes in one household and shared risk may be greater, not smaller.

Second, institutional incentives has also potential to alter the current status-quo Even in many European welfare states, tax and transfer systems reward marriage and childbearing particularly at the lower end of the income distribution. \citet{Steiner2011} document how Germany’s joint taxation system disproportionately benefits low-income households, while \citet{DoepkeKindermann2019} show that child subsidies shift family formation decisions, especially among poorer families. In such contexts, this would be a clear and valid reason for the marginal financial gain from marrying may outweigh the costs, in sharp contrast to the U.S.

Third, cultural and ideological factors complicate this narrative and make the puzzle more interesting. Conservative or religious groups, which are often overrepresented among poorer populations, attach higher symbolic and moral value to marriage. Studies such as \citet{Glaeser2007} and \citet{Lesthaeghe2010} suggest that cultural norms can sustain higher marriage propensities among the disadvantaged, even when resources are limited, which can increase the incentive to marry.

The coexistence of these perspectives and reasons create a genuine puzzle. On one hand, the dominant U.S.-based empirical literature demonstrates that poverty depresses marriage as discussed in the class and literature. On the other hand, both theory and cross-national evidence indicate mechanisms through which poverty may actually encourage household formation especially among the most disadvantaged groups.

The motivation behind this question is also to suggest a solution for a prominent topic. Marriage remains a central institution linking family formation with fertility. Advanced economies face unprecedented demographic challenges: fertility rates below replacement, aging populations, and the looming fiscal strain of supporting growing retiree cohorts with shrinking working-age populations \citep{BloomCanningFink2010, DoepkeTertilt2016}. Understanding how poverty shapes marriage decisions is therefore crucial for evaluating demographic policy. If the poor are less likely to marry because of cost barriers, fertility-enhancing policies may need to address income insecurity and upfront expenses. If, instead, poor households marry more when fiscal incentives align, then pro-natalist policies may be especially effective when targeted toward disadvantaged groups. Either way, resolving this puzzle helps illuminate the dynamics behind aging populations and low birth rates.

\noindent \textbf{Relation to Literature.}
The project is positioned at the intersection of three literatures. First, the U.S. “marriage gap” tradition emphasizes economic instability as a deterrent to marriage \citep{Wilson1987, Murray1994, EdinKefalas2005, Autor2019}. Second, the family policy literature in Europe highlights how tax incentives and subsidies shape household formation \citep{Steiner2011, EllwoodJencks2004, DoepkeKindermann2019}. Third, cultural sociology emphasizes conservative values as marriage-promoting, even under economic strain \citep{Cherlin2004, Lesthaeghe2010}. Each perspective captures part of the story, but none integrate them into a framework that explains when poverty reduces and when it increases marriage. This project addresses that gap.

\section*{Contribution}

This project advances the literature in three ways:

\begin{enumerate}
\item \textbf{Theoretical.} Develop a unified framework in which marriage has dual roles: a costly commitment good and a household insurance mechanism. This framing clarifies the conditions under which poverty suppresses marriage (when costs dominate) versus encourages it (when risk-sharing, subsidies, or norms dominate).
\item \textbf{Empirical.} Conduct a comparative study of the U.S. and Germany, with extensions to other OECD countries, to test how the poverty–marriage relationship depends on fiscal and cultural environments. Identification strategies will exploit quasi-experimental policy reforms such as U.S. Earned Income Tax Credit expansions and Germany’s Elterngeld parental leave reform.
\item \textbf{Policy relevance.} Connect the micro-level determinants of marriage among the poor to macro-level challenges of aging and fertility decline. The results will inform whether family subsidies and tax reforms can effectively stimulate household formation in disadvantaged groups, thereby contributing to demographic sustainability.
\end{enumerate}

\section*{Data}

The empirical strategy relies on rich microdata and cross-country sources:

\begin{itemize}
\item \textbf{United States.} Panel Study of Income Dynamics (PSID), Current Population Survey (CPS), and Survey of Income and Program Participation (SIPP) provide longitudinal data on marriage, fertility, income, and program participation.
\item \textbf{Germany.} The German Socio-Economic Panel (SOEP) offers detailed household-level information on marriage and family policy exposure, including subsidies and tax treatment.
\item \textbf{Cross-country comparison.} OECD Family Database, World Values Survey (WVS), and European Social Survey (ESS) capture institutional variation and cultural attitudes toward marriage.
\item \textbf{Policy reforms.} U.S. Earned Income Tax Credit (EITC) expansions in the 1990s, Germany’s 2007 Elterngeld parental leave reform, and child allowance reforms in Nordic countries provide quasi-experimental variation to study causal effects of income and subsidies on marriage.
\end{itemize}

These sources allow the project to test systematically whether poverty deters or encourages marriage, depending on institutional and cultural environments. The findings will deepen our understanding of how inequality interacts with family formation and provide critical insights for demographic policy in the context of aging populations.
\newpage
\bibliographystyle{apalike}
\bibliography{reference3}

\end{document}


